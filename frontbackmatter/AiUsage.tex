%*******************************************************
% Usage of Generative AI
%*******************************************************
\refstepcounter{dummy}
\pdfbookmark[0]{Usage of Generative AI}{Usage of Generative AI}
\chapter*{Usage of Generative AI}
\thispagestyle{empty}
OpenAI GPT-5 and GPT-5.2 (OpenAI, 2025) \cite{openai_chatgpt_2023} was used in order to aid in the creation of this thesis as described below.

\begin{itemize}
	\item In the literature review to help expand search terms and improve search quality by considering additional / similar keywords leading to better search result in Google Scholar and other search engines as well as by leveraging the "Deep Research" functionality which conducts a multi-step research on the internet.
	\item In order to assist in the writing process. Textual information was not generated with the assistance of AI. Rather, it was used to proofread, give feedback, and support in giving the paper an academic writing style.
	\item In order to summarize meeting notes, ideas, and protocols while organizing the action research.
\end{itemize}