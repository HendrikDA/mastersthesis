%********************************************************************
% Appendix
%*******************************************************
% % If problems with the headers: get headings in appendix etc. right
\markboth{\spacedlowsmallcaps{Appendix}}{\spacedlowsmallcaps{Appendix}}
\chapter{Instruction Manual for \prototype}
\label{appendix:ch:instructions}
\prototype{} is the name of the prototype enterprise architecture chatbot. Follow these instructions to download \prototype{}, set it up, and learn how to use it.

When executing the commands from the terminal, always execute them from the root folder of the project. The best results are achieved by executing each command individually (i.e. copy-paste the commands individually line-for-line instead of the entire block).

\section{Downloading \prototype{}}
\label{appendix:instructions:downloading}
In order to get \prototype{} up and running, a few prerequisites will have to be fulfilled.

The first is to download the application code for \prototype{} from GitHub: \href{https://github.com/HendrikDA/masuta-ea-chatbot-prototype}{https://github.com/HendrikDA/masuta-ea-chatbot-prototype}. Download the entire folder as a .zip file (under Code\textrightarrow Download ZIP) and unpack it locally.

The second prerequisite is to download Docker and Docker Compose. Instructions on how to set this up on various operating systems can be found under this link: \href{https://docs.docker.com/compose/install/}{https://docs.docker.com/compose/install/}.

An optional prerequisite is to download neo4j Desktop. This allows the user to inspect the graph database within an interactive desktop application. Neo4j Desktop can be found here: \href{https://neo4j.com/download/}{https://neo4j.com/download/}. However, this is not mandatory and the databases may also be inspected via neo4j's web-application.

% First time setup
\section{First Time Setup}
\label{appendix:instructions:first}

% Environment Variables
\subsection{Setting Environment Variables}
\label{appendix:env_vars}
Before being able to run any containers, a few environment variables must be set. This requires creating and editing \texttt{.env} files throughout the project. It may be required to use a code editor to execute this step. All \texttt{.env} files are prepared via the adjacent \texttt{.env.example} file of the individual folder. The following steps explain how to create the required \texttt{.env} files per sub-folder:
\begin{itemize}
	\item \textbf{Project Root Folder}: Within the root folder of the project, simply duplicate the \texttt{.env.example} file and rename the duplicated file to \texttt{.env}. Ensure it is in the root folder of the project. No values in the \texttt{.env} file need to be changed for the prototype to work.
	\item \textbf{Frontend}: Within the frontend folder, simply duplicate the \texttt{.env.example} file and rename the duplicated file to \texttt{.env}. Ensure it is in the root folder of the ./frontend/ folder. No values in the \texttt{.env} file need to be changed for the prototype to work.
	\item \textbf{MCP Backend}: Within the mcp-backend folder of the project, simply duplicate the \texttt{.env.example} file and rename the duplicated file to \texttt{.env}. Ensure it is in the root folder of the ./mcp-backend/ folder. Within the \texttt{.env} file, the value  for \textbf{OPENAI\_API\_KEY} must be set. It may require that an API key is generated within OpenAI's API platform. All other values may remain as they are for the prototype.
\end{itemize}

Once these environment variables are set, the databases may be set up.

% Preparing Database Backups
\subsection{Preparing the Databases}
\label{appendix:db_backups}
Before being able to set up the databases with the example data, the backup files need to be downloaded and placed into the correct folders. They can be found under the following links:

\begin{itemize}
  \item Textbook Data: \href{https://bscw.frankfurt-university.de/EduRes/bscw/bscw.cgi/d21657796/neo4j-2025-11-16T12-53-54-fde218db.backup}{neo4j-2025-11-16T12-53-54-fde218db.backup}
  \item SpeedParcel Data: \href{https://bscw.frankfurt-university.de/EduRes/bscw/bscw.cgi/d21814333/neo4j-2025-12-10T21-44-58-fde218db.backup}{neo4j-2025-12-10T21-44-58-fde218db.backup}
\end{itemize}

If the above files are not accessible, please reach out to the author or advisors of this thesis.

Once both files are downloaded they are ready to be placed within the project's folders. The Textbok Data backup file must be placed within the folder \texttt{./textbook-data/}. The SpeedParcel Data backup file must be placed under \texttt{./speedparcel-data/}. This ensures that the backup files are present and in the correct folders when running the Docker commands to set up the databases in the next step.

% Running Containers
\subsection{Setting Up the Databases}
\label{appendix:database_setup}
If \prototype{} is being run for the first time, then two containers need to be executed once so that they populate both databases with the default data. From within the root folder of the project, execute the following commands

% Setting up the docker containers when running them for the first time
\begin{lstlisting}[style=dockerstyle, label={lst:first_time_setup}]
docker compose --profile speedparcel-restore run --rm speedparcel-restore
docker compose --profile textbook-restore  run --rm textbook-restore
\end{lstlisting}

\textbf{Windows}: If the executing system runs on Windows, this step requires the use of WSL (Windows Subsystem for Linux) and to set the HOME variable via \texttt{SET HOME=\textbackslash{}C:\textbackslash{}Users\textbackslash{}<username>}. It may be required to execute these commands from a normal terminal instead of the Windows PowerShell.

Once these commands have run through successfully, both databases are populated with the default data and are ready to be used.

% Hard Resetting all containers
\subsection{Hard-Resetting Everything}
\label{appendix:instructions:hardreset}
If at any time a hard-reset for the entire application is required, run the following commands:
% Resetting the docker containers
\begin{lstlisting}[style=dockerstyle, label={lst:reset_docker}]
docker compose down
rm -rf $HOME/neo4j/data
rm -rf $HOME/neo4j_empty/data
\end{lstlisting}

After resetting the application, restart with the first time setup.

% Running the application
\section{Running the Application}
\label{appendix:instructions:running}
Running \prototype{} is as simple as running the following command:
% Run Masuta
\begin{lstlisting}[style=dockerstyle, label={lst:start_masuta}]
docker compose up
\end{lstlisting}

When running this command for the first time, it may take a while as it will have to download dependencies.

With this command, four containers are started.
\begin{itemize}
	\item The SpeedParcel neo4j database filled with example data and the textbook
	\item The playground neo4j database filled with only the textbook
	\item The MCP-Backend which is the backend node.js which by default runs under http://localhost:4000
	\item The frontend which by default runs under http://localhost:3000/
\end{itemize}

If Docker does not automatically open the frontend in the browser, navigate to it under \href{http://localhost:3000/}{http://localhost:3000/}. Everything is now ready and \prototype{} may be used.


% Feature Overview
\section{Feature Overview}
\label{appendix:instructions:features}
This section details the features within \prototype{}.

% Input and Chat History
\subsection{Input Field and Chat History}
\label{appendix:instructions:input_and_chat}
At the bottom of the application is an input field where the user may prompt \prototype. Pressing enter sends the prompt. The user's input prompt then appears as a chat bubble bound to the right of the chat-area. The system then starts working on answering the prompt and displays a chat bubble bound to the left of the chat-area displaying "Thinking...". As soon as the application is done thinking, the response is displayed in the same chat bubble.

\prototype's response has a button appended to it which allows the user to copy the cypher which was used to generate the answer. The user may take this copied cypher and paste it into neo4j desktop to inspect the raw response that was used to generate the answer.

% Toggling the database
\subsection{Toggling the Database}
\label{appendix:instructions:toggle}
At the top right of the application is a toggle switch. This allows the user to toggle between the SpeedParcel database and the playground database.

% Resetting the Database
\subsection{Resetting the Playground Database}
\label{appendix:instructions:reset_playground}
Under the menu at the top left, the user is presented with the option to reset the database. This option is only available if the playground database is selected via the toggle switch. After confirming the reset in the dialog that pops up, the playground is reset to its default state with only the textbook information. The SpeedParcel database cannot be reset via the UI.

% Importing Data
\subsection{Importing Custom Data}
\label{appendix:instructions:import}
Under the menu at the top left, the user is presented with the option to import their own data. This option is only available if the playground database is selected via the toggle switch. Within the dialog that pops up, the user can add up to 10 files either by selecting them via the file system or by dragging-and-dropping them into the dialog. Only .xml files are accepted. Check section \ref{appendix:instructions:exportArchi} on how to export XML files from ArchiMate.

After clicking the import button, the files are uploaded. If the upload was successful, an alert is shown notifying the user of this. The uploaded data remains within the realms of the Docker containers and is not uploaded to any third-party sources. The data can then be queried if the database toggle is set to the playground database.

% Inspect Graph Database
\subsection{Inspect Database}
\label{appendix:instructions:inspect_database}
Under the menu at the top left, the user is presented with the option to view the graph data in neo4j browser their own data. This option is available for both the SpeedParcel and playground database. Clicking this opens another tab under \href{https://console-preview.neo4j.io/tools/query}{https://console-preview.neo4j.io/tools/query}. Here, the user can connect to the selected database. The connection string is displayed in a dialog within \prototype. Here, the user can inspect the graph structure and paste the cyphers used for the generated responses.

In order to access the neo4j console, it may be required that the user has a neo4j account.

% Chat Context
\subsection{Chat Context and Building on Previous Answers}
\label{appendix:instructions:context}
\prototype{} saves the chat history so that the user is able to build upon previous prompts. The chat history is set so that the single previous user prompt and corresponding response is saved. This means that a further question can be asked about the previous prompt or its response. The chat history's context only goes back one prompt and response.


% Exporting from ArchiMate
\section{Exporting XML Files From ArchiMate}
\label{appendix:instructions:exportArchi}
It is recommended to import XML files into \prototype{} that have been exported directly from Archi. To do so, within the Archi application, click on file \textrightarrow{} export \textrightarrow{} model to open exchange file. Save this to a location you can find later. Notice that the exported file is an XML file. This can then be imported into \prototype{} as described in section \ref{appendix:instructions:import}.


% Using a Different LLM
\section{Using a Different LLM}
\label{appendix:instructions:llm}
Although \prototype{} was implemented with the intention of using an LLM from Open AI, it is technically possible to use a different LLM. To do so, the source code needs to be edited in x different locations as follows:

\begin{itemize}
	\item ./mcp-backend/.env: Set the variable of the OPENAI\_API\_KEY to the API key of whatever new LLM you wish to use.
	\item ./mcp-backend/src/server.ts: The root server file needs to have all calls toward Open AI replaced with calls to the new LLM. These can be found in the functions \texttt{nlToCypher()} and \texttt{explainResult()}. The structure of the calls made and the way they are subsequently handled may have to be refactored as well. It must also be ensured that the structure of the response to the frontend remains the same, otherwise a refactoring on the frontend will also be necessary.
\end{itemize}

% Example Questions
\section{Example Questions}
\label{appendix:instructions:questions}
Questions can be broken down into three categories:
\begin{enumerate}
	\item Category 1: Textbook questions about enterprise architecture management
	\item Category 2: Basic information about the enterprise architecture data
	\item Category 3: Questions that span both category 1 and 2 and are more complex than questions in category 2. These require more in-depth information about the data as well as how to deal with it as an enterprise architect.
\end{enumerate}

The following list of questions serve as examples that \prototype{} can handle. These simply serve as ideas to test the system - the user may input their own thought up questions.
 
\begin{enumerate}
   \item Questions in Category 1
   \begin{itemize}
     \item What are the schools of enterprise architecture management?
     \item How does enterprise architecture relate to town planning?
   \end{itemize}
    \item Questions in Category 2
   \begin{itemize}
     \item How many capabilities are supported?
     \item How many applications are in use?
     \item Which application supports the most capabilities?
     \item If I remove application x, which capabilities will be affected?
   \end{itemize}
    \item Questions in Category 3
   \begin{itemize}
     \item We plan on removing application x. What do we have to look out for when doing so?
     \item We wish to implement a new application x which covers the capabilities x, y, and z. Which existing applications may become redundant?
     \item Which capabilities currently rely on single-point-of-failure applications? How should we deal with this?
     \item Which capabilities are over-supported by multiple applications?
     \item What interfaces does application x have?
     \item How big of a task is it to migrate from application x to application y?
   \end{itemize}
\end{enumerate}

This concludes the instruction manual for \prototype{}. Beyond the example questions, users are encouraged to prompt \prototype{} to explore the example data provided in SpeedParcel or to analyze their own proprietary data after importing it, in order to gain insights into their enterprise architecture.

