\chapter{Introduction}
\label{ch:intro}

% Intro sentence on how EAM can benefit from LLMs
In the field of  xyz. Define that we are building a RAG system which will be referred to as a chatbot. it is not an agent, as it is not executing anything.

% Thesis goal
Goal of the thesis: this does not have to be a central question, but rather a concrete goal (as discussed on 10.02):  definier einfach ein ziel, dass für einen konkreten anwendungsbereich ein AI chatbot einem Junior Berater unterstützt mit der speicherung von fachspezifischen wissen um später explorativ zu evaluieren.

% Thesis Structure
This thesis is structured as follows. Chapter \ref{ch:background} defines the key concepts used throughout the research. Chapter \ref{ch:SOTA} gives an overview of current research, implementations within the domain, and how this research fits into the existing approaches. Chapter \ref{ch:actionResearch} describes what the Action Research method is, while the subsequent chapters \ref{ch:cycle1}, \ref{ch:cycle2}, \ref{ch:cycle3}, and \ref{ch:cycle4} describe the details of each iteration of the development. The Action Research is concluded in the final review found in chapter \ref{ch:finalReview}. Building on the action research, chapter \ref{ch:implementation} explains the final prototype with all of its components and technical descriptions. Finally, chapter \ref{ch:conclusion} presents the conclusion and possibilities for future work.

This thesis presupposes an understanding of general terms in the realm of application development and enterprise architecture management. Understanding at an intermediate level is sufficient, as all of the key concepts are explained in detail in the next chapter.