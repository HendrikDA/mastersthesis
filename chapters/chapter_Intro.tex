\chapter{Introduction}
\label{ch:intro}


%
% Motivation
%
\section{Motivation and Thesis Question}
\label{sec:intro:motivation}
In the field of 
define that we are building a RAG system which will be referred to as a chatbot. it is not an agent, as it is not executing anything.

\section{Research Method}
\label{sec:researchMethod}
Action Research was applied in order to gain scientific value out of the developed prototype.  The advantage of this research method is that it is very supportive of the development process for information systems. According to Baskerville (1999), all types of Action Research have the following four characteristics in common: An orientation towards developing, a focus on a specific problem, an iterative process, and a collaboration amongst participants. This is applicable to the work at hand because a prototype is being developed for a specific problem. The development cycle is conducted iteratively and collaboratively with different stakeholders. More details on this are described in chapter \ref{sec:actionResearch} \cite{baskerville1999investigating}


This thesis is structured as follows. Chapter \ref{ch:background} defines the key concepts used throughout the research. Chapter \ref{ch:SOTA} gives an overview of current research, implementations within the domain, and how this research fits into the existing approaches. Chapter \ref{ch:method} describes how the Action Research method was applied and how the development iterations were conducted. Building on this, chapter \ref{ch:implementation} explains the final prototype with all of its components. Chapter \ref{ch:experiments} describes how the final prototype was tested and evaluated. Chapter \ref{ch:Discussion} discusses the implementation, what went well, and what could be improved. Finally, chapter \ref{ch:conclusion} presents the conclusion and possibilities for future work.