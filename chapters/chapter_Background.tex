\chapter{Terminology and Technology}
\label{ch:background}
This chapter goes in detail on the terminology and technology that will be relevant for the reader to have a foundational understanding of the rest of this thesis. Later chapters will build upon these concepts and pieces of technology.

\section{Terminology}
\label{sec:background:terminology}

\subsection{Enterprise Architecture Management}
\label{sub:background:eam}
Enterprise Architecture Management (EAM) can be summarized as being the bridge between the business and IT departments of an enterprise. The goal is to implement information technology that is aligned with the business needs of the company. This is in contrast to the IT department implementing information technology for the sake of implementing information technology, which people in IT are often fond of doing \cite{ahmed2017motivating}. An unwanted situation would then be when the IT department falls into a siloed way of thinking where they are decoupled from the rest of the company. EAM helps to ensure that the implemented information technology is achieving the right things, namely supporting the business capabilities and processes. \cite[pg. 2-3]{jung2021masterclass}

Enterprise Architecture can benefit a company in various ways. 

% How can this be achieved? what does EAM do in detail?

Diese Quelle hierfür vielleicht verwenden: \cite{jung2019purpose} Die Quelle hat vor allem eine Menge Fragen aus der echten welt, was EAs sich im alltag fragen. diese liste kann als referenz dienen für meinen copilot. Diese Quelle hilft da vielleicht auch nochmal um in die Tiefe zu gehen. \cite{castro2021towards}

\subsection{Enterprise Architect}
\label{sub:background:ea}

A common challenge for an EA is dealing with the heterogenous nature of an application landscape \cite[pg. 6]{jung2021masterclass}

Erwähnen, dass eine Herausforderung des EAs es ist, dass die vorzunehmenden Änderungen zwar von einem high-level POV einfach aussehen, aber in den details viele herausforderungen stecken. zB stakeholder management (jede Applikation hat einen eigenen verantwortlichen, viele schnittstellen der applikationen, etc.)

\subsection{Capabilities}
\label{sub:background:capabilities}

\subsection{Value Streams}
\label{sub:background:valuestreams}

% All main types of architecture diagrams, such as application landscapes, capability maps, CS-Matrix, etc.
\subsection{Architecture Diagrams}

\subsubsection{Application Landscape}
\label{subsub:background:landscape}

\subsubsection{Capability Support Matrix}
\label{subsub:background:cap-support-matrix}




\subsection{Large Language Models}
\label{sub:background:llms}

LLMs are capable of supporting in language-related tasks where text needs to be generated, translated, summarized, analysed, or questions answered \cite{hadi2023large}.

% Architecture of LLMs

\subsection{Graph Database}
\label{sub:background:graphdb}


\subsection{Retrieval Augmented Generation}
\label{sub:background:rag}

\subsection{neo4j}
\label{sub:background:neo4j}

\subsection{Model Context Protocol}
\label{sub:background:mcp}
Also explain (maybe in the SOTA) how it is able to understand the schema of the graphdatabase (by calling get-schema	when initializing the MCP it knows how my graph database is setup). 
