\chapter{Terminology and Technology}
\label{ch:background}
This chapter goes in detail on the terminology and technology that will be relevant for the reader to have a foundational understanding of the rest of this thesis. Later chapters will build upon these concepts and pieces of technology.

\section{Terminology}
\label{sec:background:terminology}

\subsection{Enterprise Architecture Management}
\label{sub:background:eam}
Enterprise Architecture Management (EAM) can be summarized as being the bridge between the business and IT departments of an enterprise. The goal is to implement information technology that is aligned with the business needs of the company. This is in contrast to the IT department implementing information technology for the sake of implementing information technology, which people in IT are often fond of doing \cite{ahmed2017motivating}. An unwanted situation would then be when the IT department falls into a siloed way of thinking where they are decoupled from the rest of the company. EAM helps to ensure that the implemented information technology is achieving the right things, namely supporting the business capabilities and processes. \cite[pg. 2-3]{jung2021masterclass}

Enterprise Architecture can benefit a company in various ways. 

% How can this be achieved? what does EAM do in detail?

\subsection{Enterprise Architect}
\label{sub:background:ea}



A common challenge for an EA is dealing with the heterogenous nature of an application landscape \cite[pg. 6]{jung2021masterclass}

\subsection{Application Landscape}
\label{sub:background:landscape}

\subsection{Capability Support Matrix}
\label{sub:background:cap-support-matrix}

\subsection{Capabilities}
\label{sub:background:capabilities}

\subsection{Value Streams}
\label{sub:background:valuestreams}


\subsection{Large Language Models}
\label{sub:background:llms}

LLMs are capable of supporting in language-related tasks where text needs to be generated, translated, summarized, analysed, or questions answered \cite{hadi2023large}.

% Architecture of LLMs

\subsection{Graph Database}
\label{sub:background:graphdb}


\subsection{Retrieval Augmented Generation}
\label{sub:background:rag}


% Technology
\section{Technology}
\label{sec:background:technology}

\subsection{neo4j}
\label{sub:background:neo4j}

