\chapter{Conclusion and Future Work}
\label{ch:conclusion}
Todo

% Conclusion
\section{Conclusion}
\label{sub:conclusion:conclusion}

main learnings include
\begin{itemize}
	\item Pre-querying the database for its schema so that cyphers can be generated dynamically
\end{itemize}


% Future Work
\section{Future Work}
\label{sub:conclusion:futureWork}
Some ideas for future work: how to improve the challenges mentioned. what other areas this could find utility in. how could access management be handled? e.g. if the database contains customer-information that not every user should be able to see, how can that be differentiated? real-world scenarios that could use my prototype and try out a pilot phase?

Vergleich mal den Pipelining prozess, den Jürgen am 27.01 vorgeschlagen hat mit n8n: \href{https://n8n.io/}. die frage analysieren, runterbrechen in einzelne schritte, und sich so das endergebnis zusammen rechnen.

Alles vom Termin am 27.01.26 nochmal durchgehen.

Downloadable dateien, die aus den openai antworten generiert werden, wären wichtig

Auch interessant wäre es, Masuta an mehr MCP angebundene Applikationen zu verbinden. Beispielsweise, dass Masuta direkt mit Archi oder Confluence verbunden ist, um die Architekturen dort zu aktualisieren anhand der Ergebnisse. zB "Please create a documentation page on how to remove application x and everything that needs to be considered from covered capabilities, alternatives as a replacement, to licensing costs".

Auch, dass man Masuta mal quantitativ messen sollte, wäre relevant. Beispielsweise, wie schnell wird eine Antwort generiert? Wie akkurat sind die antworten? Wie oft kommen fehler auf? etc.