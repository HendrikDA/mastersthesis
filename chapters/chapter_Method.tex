\chapter{Methodology}
\label{ch:method}
This chapter describes how the chatbot “\prototype{}” (pronounced “mah-soo-taa” - a phonetic adaptation of the English term "master" into Japanese katakana) was developed. It provides further insight into the applied research method and the process through which the prototype was created. This gives the reader the understanding of how the research and implementation was conducted before discussing the implementation details in chapter \ref{ch:implementation}.

%%%%
% Action Research in theory
%%%%
\section{Action Research Design}
\label{sec:action_research_design}
Action Research (AR)  is a research method which is highly applicable when developing an information system such as the one presented in this paper. The advantage of AR is that a large focus can be laid on the development of a system while still achieving an academic benefit. This allows for a very explorative approach to developing an information system.

% Why action research applies well to software development (case studies)


% What cyclical phases are
Cyclical phases are central to the concept of Action Research. Baskerville \cite{baskerville1999investigating} describes Action Research as an iterative process consisting of five steps within a single cycle. Other sources, such as Cornish et al. \cite{cornish2023participatory}, propose variations with fewer (usually three) phases within a cycle; however these models also boil down to the same concepts. Across the literature, Action Research cycles follow the same structure: planning what should be done in the new cycle, taking action, and evaluating the outcome of the completed cycle before moving on to the next one \cite{baskerville1999investigating, cornish2023participatory}.

% Steps in the cycle
The paper at hand applied a cycle using the following five steps according to Baskerville \cite{baskerville1999investigating}: diagnosing, action planning, action taking, evaluating, and specifying learning. The reason for choosing five cycles instead of three is the benefit of describing the development process in more detail. This five-step-cycle including the preliminary and subsequent steps are summarized in figure \textbf{to do create a drawing of the cycles}.

% Cycle duration
Each cycle lasted between three and four weeks. The sources used do not mention how long a single cycle should last. However, a cycle of two to three weeks were deemed as reasonable for the development of \prototype{} because status updates were held at the end of each cycle and with the given amount of time for the cycle, there was enough progress to discuss in these meetings.
% Participatory Action Research
Participatory Action Research (PAR) goes a step further in creating a more collaborative environment between the researcher and client participants. Instead of leaving the theorizing up to the researcher, new information and ideas are thought up together with the client participants, giving both parties an active role. This is beneficial because the client participants often have both theoretical and practical knowledge of the subject matter being worked on. \cite{baskerville1999investigating}

The following subsections explain the research design of the applied action research. For the exhaustive action research protocol, refer to chapter \ref{appendix:action_research_protocol} of the appendix.

%%%%
% Action Research Setup
%%%%
\section{Action Research Setup}
\label{sec:action_research_setup}

% Domain
The domain of EAM was focused on within this research. In particular, the study addressed application landscapes, business capability maps, as well as the relationships between these two architectural artifacts. These artifacts are commonly used to support documenting an enterprise's landscapes are used to align an enterprise's IT with its strategic business objectives.

Due to their structural complexity with heterogenous data sources and multiple stakeholders involved, these artifacts are often large and difficult to interpret. Maintaining an overview can be especially challenging for junior level enterprise architects. This challenge motivates the exploration of AI-supported solutions that enable conversational interaction with the architecture, rather than manually navigating the complex diagrams.

% Stakeholders
The research was conducted in close collaboration with the academic supervisor and a co-advisor. The academic supervisor gave academic guidance and supported the structuring of the research process to ensure academic relevance. The co-advisor acted as the domain expert and practitioner, contributing practical insights to ground the research in real-world relevance. The author of this thesis assumed the role of the researcher, implementing the action research cycles, validating findings, and planning subsequent steps in coordination with the other stakeholders.

% Initial Problem Statement
At the outset of the research, the general problem space was clear, but the potential solution was only vaguely defined. The co-advisor initialized the research with a vague vision of a centralized system containing all enterprise architecture information, which can be interacted with via natural language. The motivation for this was to reduce the effort required to interpret the enterprise architecture artifacts.

However, in the early stages of the project, not only were the technical details of the potential solution unclear, but also the feasibility of such a system. Early on it was mutually agreed upon that LLMs and AI would play a key role in realizing this, even though the data structures, mechanisms, storage options, and interaction patterns were still open. Consequently, the initial problem statement was intentionally formulated at a high level to provide a suitable starting point for iterative exploration. Through the iterative development cycles, this high level problem statement was refined into a concrete problem definition and technical solution.


% Constraints and Data

%%%%
% Action Research Cycles
%%%%
\section{Action Research Cylces}
\label{sec:action_research_cycles}
The above mentioned vague details of the implementation became clear during the development cycles, leading to a final architecture with a clear structure. Each cycle began with a meeting between all three stakeholders. The end of one cycle was the incubator for the next cycle and was conducted in the same meeting. Table \ref{tab:action-research-cycles} summarizes these cycles.

% Development Cycles
Todo: note in the cycles where i attemtped implementation x, which didn't work out because of reason y. any path taken should be noted.
% Cycle 1
\subsection{Cycle 1}
\label{sub:cycle1}
\textbf{Diagnosis}:  The initial problem statement lacked concrete technical formulation and the feasibility of natural language interaction with enterprise architecture data was unclear.

The co-advisor outlined typical enterprise architecture workflows and the tools used to document and interact with architectural artifacts in practice. An initial idea was propsed of an AI-based black-box system capable of ingesting architecture data and deriving its own internal representations. The feasibility as well as academic applicability of such a black-box system were critically questioned, particularly because of the limited transparency of the innern mechanisms and how to test this. As an alternative, the academic supervisor proposed an explicit knowledge-graph approach in which a knowledge-graph is built which can be used to supplement generated answers.

The key distinction between the black-box and white-box approaches is the transparency. While the black-box approach autonomously creates an internal representation of the information, the white-box approach requires the manual design and implementation of an explicit technical architecture.

These discussions were necessary in order to scope the solution space. Early solution ideas included a chat that would support enterprise architects in exploring and improving application landscapes, for example by identifying inconsistencies or incomplete application landscapes.

\textbf{Action Planning}: The first cycle aimed to develop a proof of concept that enables conversational access to a knowledge-graph consisting of enterprise architecture knowledge grounded in textbook-based domain information. It was decided that a single-agent architecture will be used, as multi-agent architectures were considered unnecessarily complex for an initial proof of concept.

At this stage, the research methodology had not yet been explicitly defined as action research, and it was initially assumed that the resulting prototype would be evaluated through an expert-interview.

\textbf{Action Taken}: An initial knowledge-graph was constructed based on content from the textbook \textit{Masterclass Enterprise Architecture Management} \cite{jung2021masterclass}. The textual content was iteratively preprocessed and transformed into graph representations, with successive refinements  applied to improve the mapping of domain concepts into nodes and relationships. After integrating the full textbook into the knowledge-graph, additional prototypical data was incorporated to enable querying the knowledge-graph of concrete architectural data.

\textbf{Evaluation}: The state of the proof of concept after the first cycle was demonstrated to both advisors and evaluated qualitatively through open discussions. Both advisors positively assessed the feasibility of the approach, and the co-advisor confirmed that an explicit knowledge-graph-based solution will be a viable direction for further development.

\textbf{Learning}: Compared to the beginning of the cycle, in which the feasibility of a centralized knowledge-graph was uncertain, the first cycle demonstrated that an explicit, white-box approach represents a practical path forward. The cycle also revealed key challenges related to the transformation of heterogeneous data into graph structures and the effective querying of such representations. Finally, it became evident that further iterations would be required to systematically address these challenges with an exploratory development process.

% Cycle 2
\subsection{Cycle 2}
\label{sub:cycle2}
\textbf{Diagnosis}: Following the initial feasibility assessment, the next challenge identified concerned extending the knowledge-graph with more information and positioning the prototype as a useful tool within the realm of enterprise architecture management. One design consideration involved separating textbook-based domain knowledge from enterprise architecture data into two distinct databases. After discussion, a single integrated database was chosen to enable direct relationships between conceptual textbook knowledge and architectural data, with the expectation of improved contextual reasoning.

The challenge of lack of real-world data to test the system was also identified. Potential datasets for further development and evaluation were explored before real-world data would be ready. The real-world data would require more time to be prepared because the co-advisor's company policy constraints that meant the data cannot be used directly and has to be sanitzed first. Instead, it was agreed upon to use data from a completed university assignment, consisting of an application landscape, business capability map, business capability support matrix, business object model, and cross-application data-flow diagram for a fictitious company named SpeedParcel.

\textbf{Action Planning}: The objective of the second cycle was to extend the knowledge-graph with additional SpeedParcel datasets. This included integrating the business capability support matrix. A second goal of this cycle was to improve the database querying method as soon as more data is available to test with.

\textbf{Action Taken}: The capability support matrix of the SpeedParcel dataset was integrated into the knowledge graph, requiring adjustments to existing query mechanisms. The idea of a Model Context Protocol (MCP) server came up during development and was implemented in order to achieve a more frictionless interaction between the frontend and the graph database. This allowed the the cyphers being used to query the database to no longer be hard-coded in the frontend, but allowed them to be generated dynamically based on the user input.

\textbf{Evaluation}: The extended prototype was again evaluated qualitatively through a demonstration by the researcher and discussions between all three stakeholders. Both advisors expressed strong interest in the approach and proposed potential directions for further development. It was confirmed by the co-advisor that the chatbot is being development in the correct direction in order to allow enterprise architects to interact with the knowledge-graph via natural language.

\textbf{Learning}:As the prototype is starting to mature, more learnings are being pulled from each phase. Particularly, the hard-coded query method that was previously implemented was proving to be insufficient for flexible interaction with the evolving graph structure. The MCP implementation allows the entire system to be more dynamic, independent of the data in the knowledge-graph.

Furthermore, three key categories of user questions were identified. The first category being conceptual questions related to enterprise architecture principles, concepts, and best practices found in textbooks. The second category being descriptive questions targeting concrete architectural elements and relationships. The third category being integrative questions combining conceptual knowledge with specific architectural contexts.

Beyond the scope of this thesis, a broader vision by the advisors emerged in which the prototype could support project planning by assessing impacts on application interfaces, systems, and stakeholders.

% Cycle 3
\subsection{Cycle 3}
\label{sub:cycle3}
\textbf{Diagnosis}: During this phase it became apparent that action research will be the most appropriate methodological framework for the project. The development process had evolved into an exploratory and iterative approach. This has been characterized by continuous development, reflection between all three stakeholders, and decision-making regarding in which direction to take the prototype. Consequently, the previously considered expert interview was no longer the methodology of choice for this thesis.

In parallel, the co-advisor began modelling real-world enterprise architecture data using the Archi tool. It was agreed upon that the real-world data would later be exported from Archi and imported into the knowledge-graph in XML format. This will enable the integration of more realistic data into the finished prototype.

Additionally, it was decided that a final meeting will be conducted at the end of the fourth cycle. This meeting would require that the prototype be finished and be executable in a local environment. This meeting will mark the end of the development phase of the prototype.

\textbf{Action Planning}: The objective of the third cycle was to further expand the information saved in the knowledge-graph by creating additional datasets from the SpeedParcel dataset. The datasets had to be replicated in Archi. This allowed them to be exported into XML format and imported into the knowledge-graph. This step aimed to prepare the system for handling more complex and structured architectural models later.

\textbf{Action Taken}: Both the business object model and the cross-application data-flow diagram were recreated in Archi, in order to test how exported XML files will behave when parsing them into the knowledge-graph. The parsing at this step was still being conducted via LLM.

During this phase, limitations of the LLM-parsing method became apparent. While the custom parser that have been used so far worked well for individual files or specific data formats, the approach reached its practical limits as data complexity increased. The parsing process became constrained by the limitations of the LLM's capabilities.

\textbf{Evaluation}: The progress of the third cycle was reviewed with both advisors. The current state of the prototype and the extended data integration were assessed positively, and the overall research trajectory was confirmed as appropriate. The end goal is in sight and is being worked towards in an appropriate manner.

\textbf{Learning}: The third cycle revealed scaling issues. While the individual per-file parsers have worked well so far, this cycle showed that its limits have been reached. Parsing the more complex files of this cycle showed that LLMs are no longer the appropriate method moving forward. This approach would also not allow for parsing without manual implementation of a parser specific for the file at hand; a problem that has to be solved before the final meeting where files will have to be parsed on site.

Another scalability issue revealed was a discussion about the real-world data. While SpeedParcel's data contained hundreds of nodes and edges, the real-world data will potentially have thousands of nodes and edges. This gap revealed that a better parsing solution and a better querying solution will be necessary before the prototype is ready for the final on site meeting.

% Cycle 4
\subsection{Cycle 4}
\label{sub:cycle4}
\textbf{Diagnosis}:

\textbf{Action Planning}: the development of a local version via docker is important for this cycle. on top of this, a universal importer needs to be created in order to import any archi-exported .xml data. also, the querying method needs to be improved to be able to universally query any data in the knowledge-graph. in total, i have to prepare everything in the prototype in such a way that our final meeting will work well.

\textbf{Action Taken}:

\textbf{Evaluation}:

\textbf{Learning}:


The main challenge of this cycle was refactoring the cyphers that get called agains the database. the reason being that the previous method for generating text-to-cypher was easy because it was only calling the speedparcel database and the schema of this was entirely known. however, during this sprint we got new data exported from architecture diagrams in archi. this data is well structured but the schema is unknown. this means that the text-to-cypher has to be able to query the graph database agnostic of any schema in it. meaning, the text-to-cypher has to be completely refactored in order to be able to reliable query the dataset with unknown contents.

The source from Wan i \ref{sub:agnostic_cypher}  \cite{wan2025prompting} explains how he created a 3-step-preprocessing in order to query the database agnostically. what i did is not 1-to-1 the same thing, but i borrowed the ideas. the main change being changing the CALL db.schema.visualization() from before to the APOC call apoc.meta.schema() which apparently returns more sensible information. that combined with the SHOW INDEXES call give a better result (i assume - i'm writing this before testing just to get my ideas out of my head lol have fun rewriting this. i wrote this in Bremen on 28.12 xoxo)


% Table summarizing the action research cycles
\begin{table}[htbp]
\centering
\caption{Overview of Action Research Cycles}
\label{tab:action-research-cycles}

\resizebox{\textwidth}{!}{%
\begin{tabular}{p{2.0cm} p{3.2cm} p{3.2cm} p{3.2cm} p{3.2cm} p{3.2cm}}
\toprule
\textbf{Cycle} &
\textbf{Diagnosis} &
\textbf{Action Planning} &
\textbf{Action Taken} &
\textbf{Evaluation} &
\textbf{Learning / Outcome} \\
\midrule
Cycle 1 & Todo & Todo & Todo & Todo & Todo  \\
Cycle 2 & Todo & Todo & Todo & Todo & Todo  \\
Cycle 3 & Todo & Todo & Todo & Todo & Todo  \\
Cycle 4 & Todo & Todo & Todo & Todo & Todo  \\
\bottomrule
\end{tabular}
}

\end{table}



%%%%
% Final Meeting
%%%%
\section{Final Meeting}
\label{sec:action_research_final_meeting}
Todo: den Termin vor Ort beschreiben
Purpose of the meeting, who participated, what was validated, what kind of feedback was collected.




% Conclusion Paragraph
As described in this chapter, it becomes clear why action research was an invaluable methodology. From unclear beginnings containing only a vague vision for a final prototype, each development cycle contributed to the final architecture being clear and goal oriented. Each phase helped to examine what was possible from a technical standpoint as well as how to move forward. This supported the explorative nature of the project.



