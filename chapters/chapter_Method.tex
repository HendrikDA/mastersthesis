\chapter{Methodology}
\label{ch:method}
This chapter describes how the chatbot "Masutā" (Japanese for "Master") was developed. It gives more insight into the applied research method and how the prototype came to be.

Sequenzdiagramm, projekt plan (timeline), herangehensweise, etc.

% Action Research
\section{Action Research}
\label{sec:actionResearch}
Action Research (AR)  is a research method which is highly applicable when developing an information system such as the one presented in this paper. The advantage of AR is that a large focus can be laid on the development of a system while still achieving an academic benefit.

Cyclical phases are central to the concept of Action Research. Baskerville \cite{baskerville1999investigating} describes Action Research as an iterative process consisting of five steps within a single cycle. Other sources, such as Cornish et al. \cite{cornish2023participatory}, propose variations with fewer phases within a cycle; however these models also boil down to the same concepts. Across the literature, Action Research cycles follow the same structure: planning what should be done in the new cycle, taking action, and evaluating the outcome of the completed cycle before moving on to the next one \cite{baskerville1999investigating, cornish2023participatory}.

The paper at hand applied a cycle using the following steps according to Baskerville \cite{baskerville1999investigating}: diagnosing, action planning, action taking, evaluating, and specifying learning. This cycle including the preliminary and subsequent steps are summarized in figure \textbf{to do create a drawing of the cycles}. Each cycle lasted between three and four weeks. The sources used do not mention how long a single cycle should last. However, a cycle of two to three weeks were deemed as reasonable for the development of Masutā.

Abgrenzen zu Participatory Action Research (was wir auch betreiben)
Participatory Action Research (PAR) goes a step further in creating a more collaborative environment between the researcher and client participants. Instead of leaving the theorizing up to the researcher, new information and ideas are thought up together with the client participants, giving both parties an active role. This is beneficial because the client participants often have both theoretical and practical knowledge of the subject matter being worked on. \cite{baskerville1999investigating}

% Preconditions
\subsection{Preconditions}
\label{sub:preconditions}

% Development Cycles
\subsection{Development Cycles}
\label{sub:devCycle}

\subsubsection{Cycle 1}
\label{sub:cycle1}

\subsubsection{Cycle 2}
\label{sub:cycle2}

\subsubsection{Cycle 3}
\label{sub:cycle3}

\subsubsection{Cycle 4}
\label{sub:cycle4}

% Data
\section{Data Used}
\label{sec:dataUsed}

% Finished Prototype
\section{Finished Prototype}
\label{sec:finishedPrototype}
Explain here, what the finished prototype is (including architecture diagram, sequence diagram, etc.). or should this be an entirely separate chapter? describing this somewhere here makes sense though before moving on to the experiments done with the prototype.
