\chapter{Current State of the Art}
\label{ch:SOTA}
Briefly explain why a literature analysis is important. Define the scope (what fields you looked at, which databases, what keywords). Define the research method and how you narrowed it down from x sources to y sources.i

%PRISMA-lite anwenden: search strings definieren (zB "EAM", "LLM"), durchsuchte datenbanken aufzählen (zB Google Scholar), wie viele Treffer ich gefunden habe und wie viele ich davon rausgefiltert habe. Prisma-flowchart zeigen? zB von 150 hits -> 30 nach dem ersten screening -> 15 am Ende übrig geblieben

% You don’t need to retell everything — the goal is to extract what is relevant to your thesis. A good mini-summary (2–5 sentences) should cover:
%Context: What problem or domain does it deal with?
%Approach: What methods, models, or tools does it use?
%Findings / Contributions: What are the main results?
%Relevance for you: Why does this matter for your thesis? (e.g., “demonstrates limitations of plain RAG approaches — motivates KG integration”)
% Template on how to summarize a paper: "Authors (Year) investigate X using Y. They found Z. For my thesis, this shows A / highlights gap B."


%
% Enterprise Architecture Management
%
\section{Enterprise Architecture Management}
\label{sec:intro:eam}
theories, digital twin efforts, EA tool landscapeStandards or frameworks (e.g., TOGAF, ArchiMate, IATA ONE Record, LeanIX).
Theoretical foundations (auch auf prozessmanagment eingehen, wie der aktuelle Prozess aussieht, wenn die Landschaft geändert werden soll)
Current tools and methods
Research prototypes in EA

Authors Jung and Fraunholz 2021 \cite{jung2021masterclass} lay foundational work from which many EAM concepts can be derived.


%
% LLMs, Conversational Agents, and RAG
%
\section{Large Language Models, Conversational Agents, and Retrieval-Augemented Generation}
\label{sec:intro:llmandrag}
strengths, hallucination issues, graph-RAG enhancements
Theoretical foundations
Current tools and methods

This paper covers how ai tools are more scalable than manual expertise analysis of things. The source is highly relevant. Look at the summary in notebookLM. 05.10.25 \cite{gu2024survey}

This 2025 paper has ideas on how changing knowledge-graphs (e.g. through updates) can be handled \cite{temporalGraphRag}. It looks at temporal data and how to handle it. This might be relevant since addressing how a changing application landscape can be handled will probably be a challenge.

This paper gives an overview on how to control the dialog sequence and also notes 4 types of dialog options for chatbots in the related works section: \cite{li2025chatsopsopguidedmctsplanning}.

This paper \cite{Zhai_2025} covers how a chatbot can support in task-planning and output generation. Might be helpful in understanding how my chatbot can tell the EA how to conduct changes in the application landscape.

This paper \cite{deng2023promptingevaluatinglargelanguage} states how proactive dialogue systems work and can be improved. It goes into detail on 3 types of dialogue systems: clarification, target-guided, and non-collaborative dialogues. All 3 of these have a certain relevance for the EAM Chatbot.

RAG: Geh darauf ein, was es für unterschiedliche Chunking methoden gibt, wie man ein Buch runterbringt, und was das alles für vor und nachteile hat. auch welche tools es gibt (neo4j) ist wichtig.

%
% Comparable Projects and Prototypes
%
\section{Comparable Projects and Prototypes}
\label{sec:intro:prototypes}
Proof-of-concepts, research prototypes, industry whitepapers, GitHub projects.

Tools like ChatEA, LeanIX AI features, or Microsoft Copilot integrations in architecture/governance.

A prototypical graph-based RAG approach for text-summarization has been created by Microsoft: y\cite{MsGraphRAGPrototype}. The accompanying paper is here: \cite{MsGraphRAGPaper}

%
% Evaluations and Limitations
%
\section{Evaluations and Limitations}
\label{sec:intro:limitations}
Studies analyzing strengths/weaknesses of RAG, embedding quality, hallucination mitigation.

Papers about user interaction with EA tools, chatbot evaluation frameworks, usability challenges

This paper \cite{evaluationFrameworkLLMs} gives a standardized method and framework for evaluating coversational AI agents.

This paper \cite{liu2023agentbench} proposes a benchmark for open-ended multi-turn conversational agents. I think this paper focuses more on evaluating agents and comparing their results, but maybe i can copy their evaluation methods and benchmarks?







