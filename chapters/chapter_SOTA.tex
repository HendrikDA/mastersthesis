\chapter{Current State of the Art}
\label{ch:SOTA}
Briefly explain why a literature analysis is important. Define the scope (what fields you looked at, which databases, what keywords).

% You don’t need to retell everything — the goal is to extract what is relevant to your thesis. A good mini-summary (2–5 sentences) should cover:
%Context: What problem or domain does it deal with?
%Approach: What methods, models, or tools does it use?
%Findings / Contributions: What are the main results?
%Relevance for you: Why does this matter for your thesis? (e.g., “demonstrates limitations of plain RAG approaches — motivates KG integration”)
% Template on how to summarize a paper: "Authors (Year) investigate X using Y. They found Z. For my thesis, this shows A / highlights gap B."


%
% Enterprise Architecture Management
%
\section{Enterprise Architecture Management}
\label{sec:intro:eam}
theories, digital twin efforts, EA tool landscapeStandards or frameworks (e.g., TOGAF, ArchiMate, IATA ONE Record, LeanIX).
Theoretical foundations
Current tools and methods
Research prototypes in EA

Authors Jung and Fraunholz 2021 \cite{jung2021masterclass} lay a foundation for xyz.


%
% State of the Art Methods and Technologies
%
\section{Large Language Models and Retrieval Augmented Generation}
\label{sec:intro:llmandrag}
strengths, hallucination issues, graph-RAG enhancements
Theoretical foundations
Current tools and methods



%
% Comparable Projects and Prototypes
%
\section{Comparable Projects and Prototypes}
\label{sec:intro:prototypes}
Proof-of-concepts, research prototypes, industry whitepapers, GitHub projects.

Tools like ChatEA, LeanIX AI features, or Microsoft Copilot integrations in architecture/governance.


%
% Evaluations and Limitations
%
\section{Evaluations and Limitations}
\label{sec:intro:limitations}
Studies analyzing strengths/weaknesses of RAG, embedding quality, hallucination mitigation.

Papers about user interaction with EA tools, chatbot evaluation frameworks, usability challenges