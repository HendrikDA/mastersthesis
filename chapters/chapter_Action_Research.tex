\chapter{Action Research}
\label{ch:actionResearch}
The next six chapters describe how the chatbot “\prototype{}” (pronounced “mah-soo-taa” - a phonetic adaptation of the English term "master" into Japanese katakana) was developed. It provides further insight into the applied research method and the undergone iterations through which the prototype was created. This gives the reader the necessary understanding of how the research and implementation was conducted before discussing the implementation details afterwards in chapter \ref{ch:implementation}.

%%%%
% Action Research in theory
%%%%
\section{Action Research Design}
\label{sec:action_research_design}
Action Research was applied in order to gain scientific value out of the development of the prototype. The advantage of this research method is that it is very supportive of the development process for information systems. According to Baskerville (1999) \cite{baskerville1999investigating}, all types of Action Research have the following four characteristics in common: An orientation towards developing, a focus on a specific problem, an iterative process, and a collaboration amongst participants. This is applicable to the work at hand because a prototype is being developed for a specific problem and in an explorative manner with the support of industry experts.

% Why action research applies well to software development (case studies)
%Todo: describe why action research applies well to software development. Jürgen hat mir hierzu ein Paper geteilt am 10.02

%Todo: add these challenges to the cycles:
%\begin{itemize}
	%\item Chunking methoden beim Einlesen des Textbuches 21.10.25
	%\item Die Datenstruktur des eingelesenen Textbuches 21.10.25
	%\item Das abfragen der GraphDB zusammen mit dem LLM
%\end{itemize}

% What cyclical phases are
Cyclical phases are central to the concept of Action Research, which contains an iterative process consisting of five steps within a single cycle. Other sources, such as from Cornish et al. (2023) \cite{cornish2023participatory}, propose variations with fewer (usually three) phases within a cycle; however these models also boil down to the same concepts. Across the literature, Action Research cycles follow the same structure: planning what should be done in the new cycle, taking action, and evaluating the outcome of the completed cycle before moving on to the next one \cite{baskerville1999investigating, cornish2023participatory}.

% Steps in the cycle
The paper at hand applied a cycle using the following five steps according to Baskerville (1999) \cite{baskerville1999investigating}: diagnosing, action planning, action taking, evaluating, and specifying learning. The reason this research applies five cycles instead of three is the benefit of describing the development process in more detail. This five-step-cycle including the preliminary and subsequent steps built the basis of the conducted Action Research.

The following list describes the goal of each of the five phases \cite{baskerville1999investigating}:
\begin{enumerate}
	\item Diagnosing: Deals with diagnosing the primary problems that require the organization to adapt.
	\item Action planning: Researchers and practitioners collaborate on what the next steps are to relieve the diagnosed pain points.
	\item Action taking: The actual implementation of the planned actions.
	\item Evaluating: The collaborative evaluation of the outcomes of the action taking phase. Evaluation includes determining if the theoretical effects of the action were realized and whether or not this relieved the problems. The output of this phase is generally the input for the next cycle.
	\item Specifying learning: The learnings of the cycle must be documented and applied to the next cycle.
\end{enumerate}

% Cycle duration
Within this research, each cycle lasted between three and four weeks. The applied sources do not mention how long a single cycle should last. However, a timeframe of two to three weeks for each cycle was deemed as a reasonable for the development of \prototype{} because status updates were held at the end of each cycle and with the given amount of time for the cycle, there was enough progress to evaluate in these status update meetings.

% Participatory Action Research
Action Research typically leaves the main theorizing up to the researcher. However, an extended form of Action Research named Participatory Action Research goes a step further in creating a more collaborative environment between the researcher and further participants. Instead of leaving the theorizing up to the researcher, new information and ideas are thought up together with the other participants, giving both parties an active role. This is beneficial because the other participants often have both theoretical and practical knowledge of the subject matter being worked on. \cite{baskerville1999investigating} From here on out, the term Participatory Action Research will be used interchangeably with Action Research.

The following subsections explain the design of the applied Action Research.

%%%%
% Action Research Setup
%%%%
\section{Action Research Setup}
\label{sec:action_research_setup}

% Domain
The domain of EAM was focused on within this research. In particular, this study addressed mostly application landscapes, business capability maps, as well as the relationships between these two architectural artifacts. These artifacts are commonly used to support documenting an enterprise's landscapes and are used to align an enterprise's IT with its strategic business objectives. % Hier könnte eventuell etwas aus dem SOTA ausgelagert werden

Due to their structural complexity with heterogenous data sources and multiple stakeholders involved, these artifacts are often large and difficult to interpret. Maintaining an overview can be especially challenging for junior level enterprise architects. This challenge motivates the exploration of AI-supported solutions that enable conversational interaction with the architecture, rather than manually navigating the complex diagrams.

% Stakeholders
The research was conducted in close collaboration with the academic supervisor and the co-advisor. The academic supervisor gave academic guidance, supported the structuring of the research process to ensure academic relevance, and also acted as an expert practitioner. The co-advisor acted as the domain expert and practitioner, contributing practical insights to ground the research with real-world relevance. The author of this thesis assumed the role of the researcher, implementing the Action Research cycles, validating findings, and planning subsequent steps in coordination with the other stakeholders.

% Initial Problem Statement
At the outset of the research, the general problem space was clear, but the potential solution was only vaguely defined. The co-advisor initialized the research with a vague vision of a centralized system containing all enterprise architecture information, which can be interacted with via natural language. The motivation for this was to reduce the effort required to interpret the enterprise architecture artifacts and to offer alternative solutions to those found on the market.

However, in the early stages of the project, not only were the technical details of the potential solution unclear, but also the feasibility of such a system. Early on it was mutually agreed upon that LLMs would play a key role in realizing this, even though the data structures, mechanisms, storage options, and interaction patterns were still open. Consequently, the initial problem statement was intentionally formulated at a high level to provide a suitable starting point for iterative exploration. Through the iterative development cycles, this high level problem statement without a planned technical concept was refined into a concrete problem definition and technical solution.

Table \ref{tab:action-research-cycles-condensed} displays a condensed version of the applied Action Research cycles, including the final review. Each cycle is described in a condense three-parts-overview, giving a summary of what each cycle accomplished. The final review is also summarized at the bottom of the table.

% Table summarizing the Action Research cycles
\begin{table}[htbp]
\centering
\caption{Overview of Action Research Cycles condensed into three parts per cycle as well as a summary of the final review workshop.}
\label{tab:action-research-cycles-condensed}
\makebox[\textwidth][c]{
\resizebox{1.25\textwidth}{!}{
\begin{tabular}{p{3cm} p{5cm} p{5cm} p{5cm}}
\toprule
\textbf{Cycle} &
\textbf{Diagnosis} &
\textbf{Action Taken} &
\textbf{Learning / Outcome} \\
\midrule
Cycle 1 & Technical feasibility. & Develop a proof of concept. & Built initial knowledge graph with textbook data. Hard-coded cyphers.\\
Cycle 2 & Need to extend data and improve queries. & Integrate SpeedParcel dataset and improve querying. & Added SpeedParcel data. Replace hard-coded cyphers with MCP-based text-to-cypher.  \\
Cycle 3 & Action Research identified as appropriate methodology. Scalability and data format issues emerged. & Expand datasets via Archi models to prepare for XML imports & Recreated SpeedParcel models in Archi. Parsed XML. \\
Cycle 4 & Over-reliance on LLMs for parsing and schema handling diagnosed as critical limiatation. & Replace LLM parsing with generalized XML parser. Refactor schema handling. Prepare executable prototype. & Implemented APOC-based XML parsing and schema retrieval. Containerized local setup. UI enhancements added. \\
Final Review &
\multicolumn{3}{p{16cm}}{
A hands-on workshop with domain experts confirmed the feasibility and value of a transparent, RAG-based EAM assistant, while also revealing limitations related to context sensitivity, schema precision, and LLM-driven Cypher generation. The prototype was comparatively assessed against an industrial AI solution using realistic EAM questions, prompt variations, and data imports. The results highlight context dependency and over-reliance on LLM reasoning as key challenges and inform concrete lessons learned and directions for future research.} \\
\bottomrule
\end{tabular}
}}
\end{table}

It is important to note that during each cycle, continuous system tests were being run in order to ensure that the system's requirements were being met. For example, when adding new data to the knowledge graph, the system was prompted to test if it is able to retrieve this new data. In most cases, it was not able to do so right after adding new data. These tests led to the system's prompt having to be continuously fine tuned in order to be able to handle the data in the knowledge graph. This was a routine step conducted regularly during each iteration of development which guided the development.

With a clear understanding of what Action Research is and how it was setup for the research at hand, the next chapter will describe the first development cycle of \prototype{}.


